
\documentclass[journal]{IEEEtran}



\usepackage{hyperref}



% *** GRAPHICS RELATED PACKAGES ***
%
\ifCLASSINFOpdf
  % \usepackage[pdftex]{graphicx}
  % declare the path(s) where your graphic files are
  % \graphicspath{{../pdf/}{../jpeg/}}
  % and their extensions so you won't have to specify these with
  % every instance of \includegraphics
  % \DeclareGraphicsExtensions{.pdf,.jpeg,.png}
\else
  % or other class option (dvipsone, dvipdf, if not using dvips). graphicx
  % will default to the driver specified in the system graphics.cfg if no
  % driver is specified.
  % \usepackage[dvips]{graphicx}
  % declare the path(s) where your graphic files are
  % \graphicspath{{../eps/}}
  % and their extensions so you won't have to specify these with
  % every instance of \includegraphics
  % \DeclareGraphicsExtensions{.eps}
\fi
% graphicx was written by David Carlisle and Sebastian Rahtz. It is
% required if you want graphics, photos, etc. graphicx.sty is already
% installed on most LaTeX systems. The latest version and documentation
% can be obtained at: 
% http://www.ctan.org/pkg/graphicx
% Another good source of documentation is "Using Imported Graphics in
% LaTeX2e" by Keith Reckdahl which can be found at:
% http://www.ctan.org/pkg/epslatex
%
% latex, and pdflatex in dvi mode, support graphics in encapsulated
% postscript (.eps) format. pdflatex in pdf mode supports graphics
% in .pdf, .jpeg, .png and .mps (metapost) formats. Users should ensure
% that all non-photo figures use a vector format (.eps, .pdf, .mps) and
% not a bitmapped formats (.jpeg, .png). The IEEE frowns on bitmapped formats
% which can result in "jaggedy"/blurry rendering of lines and letters as
% well as large increases in file sizes.
%
% You can find documentation about the pdfTeX application at:
% http://www.tug.org/applications/pdftex





% *** MATH PACKAGES ***
%
%\usepackage{amsmath}
% A popular package from the American Mathematical Society that provides
% many useful and powerful commands for dealing with mathematics.
%
% Note that the amsmath package sets \interdisplaylinepenalty to 10000
% thus preventing page breaks from occurring within multiline equations. Use:
%\interdisplaylinepenalty=2500
% after loading amsmath to restore such page breaks as IEEEtran.cls normally
% does. amsmath.sty is already installed on most LaTeX systems. The latest
% version and documentation can be obtained at:
% http://www.ctan.org/pkg/amsmath





% *** SPECIALIZED LIST PACKAGES ***
%
%\usepackage{algorithmic}
% algorithmic.sty was written by Peter Williams and Rogerio Brito.
% This package provides an algorithmic environment fo describing algorithms.
% You can use the algorithmic environment in-text or within a figure
% environment to provide for a floating algorithm. Do NOT use the algorithm
% floating environment provided by algorithm.sty (by the same authors) or
% algorithm2e.sty (by Christophe Fiorio) as the IEEE does not use dedicated
% algorithm float types and packages that provide these will not provide
% correct IEEE style captions. The latest version and documentation of
% algorithmic.sty can be obtained at:
% http://www.ctan.org/pkg/algorithms
% Also of interest may be the (relatively newer and more customizable)
% algorithmicx.sty package by Szasz Janos:
% http://www.ctan.org/pkg/algorithmicx




% *** ALIGNMENT PACKAGES ***
%
%\usepackage{array}
% Frank Mittelbach's and David Carlisle's array.sty patches and improves
% the standard LaTeX2e array and tabular environments to provide better
% appearance and additional user controls. As the default LaTeX2e table
% generation code is lacking to the point of almost being broken with
% respect to the quality of the end results, all users are strongly
% advised to use an enhanced (at the very least that provided by array.sty)
% set of table tools. array.sty is already installed on most systems. The
% latest version and documentation can be obtained at:
% http://www.ctan.org/pkg/array


% IEEEtran contains the IEEEeqnarray family of commands that can be used to
% generate multiline equations as well as matrices, tables, etc., of high
% quality.




% *** SUBFIGURE PACKAGES ***
%\ifCLASSOPTIONcompsoc
%  \usepackage[caption=false,font=normalsize,labelfont=sf,textfont=sf]{subfig}
%\else
%  \usepackage[caption=false,font=footnotesize]{subfig}
%\fi
% subfig.sty, written by Steven Douglas Cochran, is the modern replacement
% for subfigure.sty, the latter of which is no longer maintained and is
% incompatible with some LaTeX packages including fixltx2e. However,
% subfig.sty requires and automatically loads Axel Sommerfeldt's caption.sty
% which will override IEEEtran.cls' handling of captions and this will result
% in non-IEEE style figure/table captions. To prevent this problem, be sure
% and invoke subfig.sty's "caption=false" package option (available since
% subfig.sty version 1.3, 2005/06/28) as this is will preserve IEEEtran.cls
% handling of captions.
% Note that the Computer Society format requires a larger sans serif font
% than the serif footnote size font used in traditional IEEE formatting
% and thus the need to invoke different subfig.sty package options depending
% on whether compsoc mode has been enabled.
%
% The latest version and documentation of subfig.sty can be obtained at:
% http://www.ctan.org/pkg/subfig




% *** FLOAT PACKAGES ***
%
%\usepackage{fixltx2e}
% fixltx2e, the successor to the earlier fix2col.sty, was written by
% Frank Mittelbach and David Carlisle. This package corrects a few problems
% in the LaTeX2e kernel, the most notable of which is that in current
% LaTeX2e releases, the ordering of single and double column floats is not
% guaranteed to be preserved. Thus, an unpatched LaTeX2e can allow a
% single column figure to be placed prior to an earlier double column
% figure.
% Be aware that LaTeX2e kernels dated 2015 and later have fixltx2e.sty's
% corrections already built into the system in which case a warning will
% be issued if an attempt is made to load fixltx2e.sty as it is no longer
% needed.
% The latest version and documentation can be found at:
% http://www.ctan.org/pkg/fixltx2e


%\usepackage{stfloats}
% stfloats.sty was written by Sigitas Tolusis. This package gives LaTeX2e
% the ability to do double column floats at the bottom of the page as well
% as the top. (e.g., "\begin{figure*}[!b]" is not normally possible in
% LaTeX2e). It also provides a command:
%\fnbelowfloat
% to enable the placement of footnotes below bottom floats (the standard
% LaTeX2e kernel puts them above bottom floats). This is an invasive package
% which rewrites many portions of the LaTeX2e float routines. It may not work
% with other packages that modify the LaTeX2e float routines. The latest
% version and documentation can be obtained at:
% http://www.ctan.org/pkg/stfloats
% Do not use the stfloats baselinefloat ability as the IEEE does not allow
% \baselineskip to stretch. Authors submitting work to the IEEE should note
% that the IEEE rarely uses double column equations and that authors should try
% to avoid such use. Do not be tempted to use the cuted.sty or midfloat.sty
% packages (also by Sigitas Tolusis) as the IEEE does not format its papers in
% such ways.
% Do not attempt to use stfloats with fixltx2e as they are incompatible.
% Instead, use Morten Hogholm'a dblfloatfix which combines the features
% of both fixltx2e and stfloats:
%
% \usepackage{dblfloatfix}
% The latest version can be found at:
% http://www.ctan.org/pkg/dblfloatfix




%\ifCLASSOPTIONcaptionsoff
%  \usepackage[nomarkers]{endfloat}
% \let\MYoriglatexcaption\caption
% \renewcommand{\caption}[2][\relax]{\MYoriglatexcaption[#2]{#2}}
%\fi
% endfloat.sty was written by James Darrell McCauley, Jeff Goldberg and 
% Axel Sommerfeldt. This package may be useful when used in conjunction with 
% IEEEtran.cls'  captionsoff option. Some IEEE journals/societies require that
% submissions have lists of figures/tables at the end of the paper and that
% figures/tables without any captions are placed on a page by themselves at
% the end of the document. If needed, the draftcls IEEEtran class option or
% \CLASSINPUTbaselinestretch interface can be used to increase the line
% spacing as well. Be sure and use the nomarkers option of endfloat to
% prevent endfloat from "marking" where the figures would have been placed
% in the text. The two hack lines of code above are a slight modification of
% that suggested by in the endfloat docs (section 8.4.1) to ensure that
% the full captions always appear in the list of figures/tables - even if
% the user used the short optional argument of \caption[]{}.
% IEEE papers do not typically make use of \caption[]'s optional argument,
% so this should not be an issue. A similar trick can be used to disable
% captions of packages such as subfig.sty that lack options to turn off
% the subcaptions:
% For subfig.sty:
% \let\MYorigsubfloat\subfloat
% \renewcommand{\subfloat}[2][\relax]{\MYorigsubfloat[]{#2}}
% However, the above trick will not work if both optional arguments of
% the \subfloat command are used. Furthermore, there needs to be a
% description of each subfigure *somewhere* and endfloat does not add
% subfigure captions to its list of figures. Thus, the best approach is to
% avoid the use of subfigure captions (many IEEE journals avoid them anyway)
% and instead reference/explain all the subfigures within the main caption.
% The latest version of endfloat.sty and its documentation can obtained at:
% http://www.ctan.org/pkg/endfloat
%
% The IEEEtran \ifCLASSOPTIONcaptionsoff conditional can also be used
% later in the document, say, to conditionally put the References on a 
% page by themselves.




% *** PDF, URL AND HYPERLINK PACKAGES ***
%
%\usepackage{url}
% url.sty was written by Donald Arseneau. It provides better support for
% handling and breaking URLs. url.sty is already installed on most LaTeX
% systems. The latest version and documentation can be obtained at:
% http://www.ctan.org/pkg/url
% Basically, \url{my_url_here}.




% *** Do not adjust lengths that control margins, column widths, etc. ***
% *** Do not use packages that alter fonts (such as pslatex).         ***
% There should be no need to do such things with IEEEtran.cls V1.6 and later.
% (Unless specifically asked to do so by the journal or conference you plan
% to submit to, of course. )


% correct bad hyphenation here
\hyphenation{op-tical net-works semi-conduc-tor}


\begin{document}
%
% paper title
% Titles are generally capitalized except for words such as a, an, and, as,
% at, but, by, for, in, nor, of, on, or, the, to and up, which are usually
% not capitalized unless they are the first or last word of the title.
% Linebreaks \\ can be used within to get better formatting as desired.
% Do not put math or special symbols in the title.
\title{Chess AI for Predicting Opponents Move}
%
%
% author names and IEEE memberships
% note positions of commas and nonbreaking spaces ( ~ ) LaTeX will not break
% a structure at a ~ so this keeps an author's name from being broken across
% two lines.
% use \thanks{} to gain access to the first footnote area
% a separate \thanks must be used for each paragraph as LaTeX2e's \thanks
% was not built to handle multiple paragraphs
%

\author{Cooper Niebuhr, Crawford Young, Kaden Range}

% note the % following the last \IEEEmembership and also \thanks - 
% these prevent an unwanted space from occurring between the last author name
% and the end of the author line. i.e., if you had this:
% 
% \author{....lastname \thanks{...} \thanks{...} }
%                     ^------------^------------^----Do not want these spaces!
%
% a space would be appended to the last name and could cause every name on that
% line to be shifted left slightly. This is one of those "LaTeX things". For
% instance, "\textbf{A} \textbf{B}" will typeset as "A B" not "AB". To get
% "AB" then you have to do: "\textbf{A}\textbf{B}"
% \thanks is no different in this regard, so shield the last } of each \thanks
% that ends a line with a % and do not let a space in before the next \thanks.
% Spaces after \IEEEmembership other than the last one are OK (and needed) as
% you are supposed to have spaces between the names. For what it is worth,
% this is a minor point as most people would not even notice if the said evil
% space somehow managed to creep in.



% The paper headers
\markboth{COMP 6600/5600 Artificial Intelligence, Spring 2025}%
{Shell \MakeLowercase{\textit{et al.}}: Bare Demo of IEEEtran.cls for IEEE Journals}
% The only time the second header will appear is for the odd numbered pages
% after the title page when using the twoside option.
% 
% *** Note that you probably will NOT want to include the author's ***
% *** name in the headers of peer review papers.                   ***
% You can use \ifCLASSOPTIONpeerreview for conditional compilation here if
% you desire.




% If you want to put a publisher's ID mark on the page you can do it like
% this:
%\IEEEpubid{0000--0000/00\$00.00~\copyright~2015 IEEE}
% Remember, if you use this you must call \IEEEpubidadjcol in the second
% column for its text to clear the IEEEpubid mark.



% use for special paper notices
%\IEEEspecialpapernotice{(Invited Paper)}




% make the title area
\maketitle

% As a general rule, do not put math, special symbols or citations
% in the abstract or keywords.
\begin{abstract}
Provide a concise yet thorough description of your project, emphasizing the problem statement and motivation. Clearly articulate the significance of the problem and why it is worth addressing. Additionally, discuss any limitations of your approach. The main content should be limited to two pages, excluding the reference list, figures, and appendices. Ensure that all required sections (Introduction, Project Plan,, Anticipated Challenges and Mitigation Strategies, Related Work and Literature Review, and CONCLUSION)are properly included and formatted according to the provided LaTeX template.
\end{abstract}

% Note that keywords are not normally used for peerreview papers.
\begin{IEEEkeywords}
put keywords for your proposal
\end{IEEEkeywords}



\IEEEpeerreviewmaketitle




\section{Introduction}

Artificial Intelligence (AI) has made significant strides in recent years, particularly in the domain of computer vision, natural language processing, and reinforcement learning. Despite these advancements, many challenges remain in adapting AI models to complex, real-world problems where data is often noisy, unstructured, or scarce. This project aims to explore an innovative AI-based solution to address a well-defined problem within this space.

We will begin by defining the problem scope, identifying key limitations in existing approaches, and proposing a novel methodology that enhances model performance, interpretability, or efficiency. The project will be grounded in recent advancements in deep learning, with a strong focus on experimental validation and quantitative assessment.

\section{Project Plan}

A well-structured project plan is essential for success. The following key aspects will be covered in depth:

\subsection{Problem Definition and Solution Approach}
We will provide a clear and concise description of the problem, its significance, and its impact on the broader research community or industry. This section will also outline our proposed solution, detailing the theoretical foundations, architectural choices, and expected improvements over existing methodologies.

\subsection{Evaluation and Performance Assessment}
To measure the effectiveness of our solution, we will define appropriate evaluation metrics and benchmarks. Our analysis will include:
\begin{itemize}
\item Quantitative comparisons with state-of-the-art models.
\item Ablation studies to isolate the impact of individual components.
\item Robustness testing under different conditions and data distributions.
\end{itemize}

\subsection{Anticipated Challenges and Mitigation Strategies}
We anticipate encountering various obstacles, including but not limited to:
\begin{itemize}
\item Computational constraints: Strategies such as model pruning, quantization, or distributed training may be explored.
\item Data limitations: Approaches such as transfer learning, synthetic data generation, or semi-supervised learning will be considered.
\item Model interpretability: Techniques such as attention visualization, SHAP values, or explainability frameworks may be incorporated.
\end{itemize}

\subsection{Related Work and Literature Review}
A comprehensive review of existing literature will be conducted to contextualize our work. We will summarize key prior contributions, highlighting their strengths and weaknesses. Our goal is to position our work relative to prior research and justify the need for our proposed approach.

\section{Topic Selection Guidelines}

We encourage selecting projects that align with the following categories:

\subsection{Application-Based Projects}
Leveraging AI techniques in new domains can uncover novel insights and applications. Possible directions include:
\begin{itemize}
\item Developing a computer vision system for novel visual datasets.
\item Applying deep reinforcement learning to real-world control problems.
\item Enhancing AI-driven decision-making in dynamic environments.
\end{itemize}

\subsection{Methodology-Focused Projects}
Contributions to the core technical aspects of AI remain highly valuable. These can include:
\begin{itemize}
\item Proposing new architectures, loss functions, or optimization techniques.
\item Enhancing existing models for improved efficiency, robustness, or scalability.
\item Investigating novel training paradigms, such as self-supervised learning.
\end{itemize}

\subsection{Best Practices for Project Execution}

We encourage students to aim for projects that:
\begin{itemize}
\item Suggest meaningful variations of established methods, incorporating modifications inspired by recent publications.
\item Adapt existing models to solve a different problem or work in a new domain.
\item Focus on experimental validation, ensuring that results are reproducible and well-documented.
\end{itemize}

Conversely, we advise against:
\begin{itemize}
\item Projects that primarily involve large-scale data collection with tedious labeling.
\item Minor modifications to existing models without substantive contributions.
\item Superficial implementations that lack rigorous evaluation.
\item Using popular, well-established projects such as YOLO and segmentation as the primary focus of the AI project without substantial modifications or novel contributions.
\end{itemize}

\section{Template Requirements}

To maintain consistency and clarity across all submissions, the following formatting guidelines must be adhered to:
\begin{itemize}
\item Use the official LaTeX project template provided.
\item Ensure proper structuring of sections, subsections, and references.
\item Figures and tables should be properly labeled and captioned.
\item Maintain consistent citation formatting using \textbackslash cite, e.g., \cite{goodfellow2016deep}.
\item The final report should be submitted as a compiled PDF document.
\end{itemize}

Non-compliance with the template requirements may result in deductions or the need for resubmission.

\section{Additional Recommendations}

To enhance the quality and clarity of the project, we suggest the following:
\begin{itemize}
\item Learn LaTeX for professional document preparation. A quick guide is available \href{https://www.overleaf.com/learn/latex/Learn_LaTeX_in_30_minutes}{here}.
\item Incorporate visual aids, such as figures and diagrams, to better illustrate key concepts.
\item Properly cite all references using \textbackslash cite, e.g., \cite{goodfellow2016deep}, to acknowledge prior work and provide context.
\item Maintain clear and structured documentation to facilitate future reproducibility and extendability of your work.
\end{itemize}

\section{Conclusion}

This project serves as an opportunity to engage in cutting-edge AI research while developing technical and analytical skills. By carefully selecting a meaningful problem, proposing an innovative solution, and rigorously evaluating results, we aim to contribute valuable insights to the AI community. We encourage thoughtful planning, diligent experimentation, and critical analysis throughout the project lifecycle.






\bibliographystyle{IEEEtran}
\bibliography{ref}

\end{document}


